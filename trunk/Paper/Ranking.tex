\documentclass[a4paper]{book}

%pachete folosite
\usepackage{graphicx}
\usepackage{fullpage}
\usepackage{verbatim}
\usepackage{url}
%pachete folosite

%calea catre diagrame
\graphicspath{{diagrams/}}
%calea catre diagrame

%pentru diacritice
\newcommand{\myqq}{''}
\catcode`\'=13
\newcommand{'}[1]{\ifmmode {}^\prime \noexpand#1%
\else\ifx#1a\u{a}% 
\else\ifx#1i\^{i}%
\else\ifx#1s\c{s}% 
\else\ifx#1t\c{t}%
\else\ifx#1j\^{a}%
\else\ifx#1A\u{A}% 
\else\ifx#1I\^{I}%
\else\ifx#1S\c{S}% 
\else\ifx#1T\c{T}%
\else\ifx#1J\^{A}%
\else\ifx#1'\myqq%
	\fi\fi\fi\fi\fi\fi\fi\fi\fi\fi\fi%
\fi}
%pentru diacritice

%titluri in romana 
\def\contentsname{Cuprins}
\renewcommand{\figurename}{Figura}
\renewcommand{\tablename}{Tabelul}
\renewcommand{\chaptername}{Capitolul}
\renewcommand{\bibname}{Bibliografie}

\newtheorem{thm}{Teorem'a}
%titluri in romana


\begin{document}


%linie orizontala
\newcommand{\HRule}{\rule{\linewidth}{0.5mm}}
%linie orizontala

%data in format romanesc
\def\today{\space  \ifcase \month
\or IANUARIE\or FEBRUARIE\or mARTIE\or APRILIE\or MAI\or IUNIE\or
IULIE\or AUGUST\or SEPTEMBRIE\or OCTOMBRIE\or NOIEMBRIE\or DECEMBRIE \fi
\space  \number\year}
%data in format romanesc

%pagina de titlu
\begin{titlepage}
\begin{center}
  
%facultatea
\textsc{\LARGE Universitatea Bucure'sti}\\
\textsc{\LARGE Facultatea de matematic'a 'si informatic'a}\\[0.5cm]
\textsc{\Large Specializarea Inteligen't'a Artificial'a}\\[1.5cm]
%facultatea
 

\textsc{\Large Lucrare de diserta'tie}\\[3.0cm]
 
 
%Titlu
\HRule \\[0.5cm]
{ \huge \bfseries Metode de ierarhizare 'in Reg'asirea Informa'tiei}\\[0.4cm]
\HRule \\[4.5cm]
%Titlu
 
%autor si coordonator
\begin{minipage}{0.4\textwidth}
\begin{flushleft} \large
\emph{Autor:}\\
C'alin \textsc{Avas'ilc'ai}
\end{flushleft}
\end{minipage}
\begin{minipage}{0.4\textwidth}
\begin{flushright} \large
\emph{Coordonator:} \\
Lect. Dr. Marius \textsc{Popescu}
\end{flushright}
\end{minipage}
%autor si coordonator
 
\vfill
 
%data curenta
{\large \today}
%data curenta
 
\end{center}
\end{titlepage}
%pagina de titlu

\tableofcontents

\chapter{Introducere}
Singtagma \emph{reg'asirea informa'tiei}(RI) are un spectru larg de in'telesuri. Scoaterea unui card de credit din portofel pentru a completa num'arul cardului 'intr-un formular este o form'a de reg'asire a informa'tiei. Dar, ca domeniu de studii academice, \emph{reg'asirea informa'tiei} poate fi definita astfel\cite{MAN}:
\begin{quote}
Reg'asirea informa'tiei se refer'a la g'asirea de material (de obicei documente) de natur'a nestructurat'a (de obicei text) care satisface
o nevoie de informa'tie, din colectii mari de date (eventual stocate pe calculatoare).
\end{quote}

Reg'asirea informa'tiei obi'snuia s'a fie o activitate practicata de ca'tiva oameni cum ar fi bibliotecarii. Acum lumea s-a schimbat 'si sute de milioane de oameni folosesc sisteme de reg'asire a informa'tiei precum motoate de c'autare web 'in fiecare zi. Reg'asirea informatiei devine rapid forma dominant'a de accesare a informatiei 'intrec'jnd modul tradi'tional de c'autare de tip "baze de date" (c'autarea dupa un num'ar de identificare).

RI este interdisciplinar'a, bazat'a pe informatic'a, matematic'a, 'stiin'ta informa'tiei, arhitectura informa\-'tiei, psihologie cognitiv'a, lingvistic'a, statistic'a, etc. Sisteme automate de RI au fost folosite pentru a reduce ceea ce se numeste "supra'inc'arcarea de informa'tie". Multe universit'a'ti 'si libr'arii publice folosesc sisteme RI pentru a facilita accesul la c'ar'ti, jurnale 'si alte tipuri de documente. Cele mai folosite aplicatii RI sunt motoarele de c'autare web.

\section{Scurt istoric}
Ideea de a folosi calculatoarele pentru a c'auta "buc'ati" de informatie a fost popularizat'a 'in articolul \emph{As We May Think} de c'atre \emph{Vannevar Bush} 'in anul 1945\cite{IRWIKI}. Primele sisteme automate de reg'asire a informa'tiei au fost introduse 'in anii 1950 si 1960. P'jn'a in 1970 cateva tehnici au fost testate cu succes pe corpusuri mici de text cum ar fi colec'tia \emph{Cranfield} (c'jteva mii de documente). Sisteme mari de RI cum ar fi \emph{Lockheed Dialog system}, au fost date 'in folosin't'a la 'inceputul anilor 1970.

'In 1992 Departamentul de Ap'arare al Statelor Unite 'impreun'a cu Institutul Na'tional de Standarde 'si Tehnologie (NIST) au sponsorizat \emph{TREC} (\textbf{T}ext \textbf{RE}trieval \textbf{C}onference) ca parte din programul TIPSTER. Scopul a fost s'a se asigure comunita'tii RI infrastructura necesar'a pentru testarea 'si evaluarea metodelor de reg'asire a textului pe colectii foarte mari. Acest program a ac'tionat ca un catalizator 'in ceea ce prive'ste cercetarea metodelor RI care scaleaza la colectii foarte mari. Apari'tia motoarelor de c'autare pe web a adus o nevoie si mai mare de sisteme RI care pot face fa'ta unor colec'tii imense de date.

\section{Descrierea procesului}
Un proces de reg'asire a informa'tiei 'incepe c'jnd un utilizator introduce o interogare (\emph{query}) 'in sistem. Interog'arile sunt formul'ari formale ale \emph{nevoilor de informa'tie}, de exemplu, cuvinte scrise 'in caseta de c'autare a unui motor de c'autare web. 'In RI o interogare nu identific'a 'in mod unic un obiect din colec'tia de obiecte. 'In schimb, mai multe obiecte pot fi considerate ca r'aspunsuri la interogare, eventual cu grade de relevan't'a diferite.

Un obiect este o entitate stocat'a 'in sistemul RI. 'In func'tie de aplic'tie, obiectele pot fi texte, documente, imagini, videoclipuri, etc. 'In general, documentele propriu-zise nu sunt 'tinute 'in sistem ci sunt reprezentate de surogate(rezultate 'in urma proces'arii documentelor) si metadate.

Majoritatea sistemelor RI calculeaz'a un scor numeric ce reprezint'a cat de bine se potriveste un obiect cu interogarea utilizatorului 'si apoi ierarhizeaz'a obiectele 'in func'tie de aceast'a valoare. Cele mai bune obiecte sunt apoi afi'sate utilizatorului.

\section{Un exemplu}
%MANNING boolean retrieval 1.1???

\chapter{Reg'asirea informa'tiei - generalit'a'ti}
%TODO: rethink this
\section{Indexare}
\section{Cautare}
\section{Reg'asirea boolean'a 'si ad-hoc}

\chapter{Evaluarea unui sistem RI}

\chapter{Metode de ierarhizare}
\section{Modelul spa'tiului vectorial (VSM)}
\subsection{Similaritatea cosinus 'si td-idf}
\section{Euristici pentru eficientizare}
\section{Cluster pruning}
\section{Modelul axiomatic}
\subsection{F2-EXP}
\section{Modelul probabilistic}
Dac'a s-ar cunoa'ste relevan'ta unui subset de documente, s-ar putea estima probabilitatea apari'tiei unui termen \emph{t} 'intr-un document relevant $P(t|R=1)$ 'si, ca urmare, acesta ar putea reprezenta baza unui clasificator care decide dac'a un document este relevant sau nu.

Utilizatorii 'incep cu \emph{nevoi de informa'tie} pe care le transform'a 'in \emph{interog'ari}. 'In mod similar, documentele sunt transformate 'in \emph{reprezent'ari de documente} care difer'a de primele cel pu'tin prin felul 'in care textul este 'imp'ar'tit 'in token-i. Baz'jndu-se pe aceste dou'a reprezent'ari, un sistem 'incearc'a s'a determine c'jt de bine satisfac documentele nevoile de informa'tii. 'In modelul Boolean sau VSM, d'jndu-se numai o interogare, pentru un sistem RI nevoia de informa'tie este incert'a. D'jndu-se interogarea 'si reprezentarea documentelor, un sistem trebuie s'a "ghiceasc'a" daca un document are con'tinut relevant pentru respectiva nevoie de informa'tie. Teoria probabilit'a'tilor pune la dispozi'tie o funda'tie de principii potrivite pentru ra'tionament 'in situa'tii incerte. Aceste principii pot fi exploatate pentru a estima c'jt de probabil este ca un document sa fie relevant pentru o nevoie de informa'tie.

Exist'a mai multe posibile modele probabilistice de reg'asire. 'In continuare voi discuta despre \emph{principiul probabilistic de ierarhizare} 'si despre \emph{modelul binar de independen't'a}, care a fost primul model probabilistic de reg'asire. 'In final voi prezenta 'si sistemul de ponderare \emph{Okapi BM25}, care a avut un succes destul de mare 'in practic'a.

'In acest context, este util conceptul de 'sanse (\emph{odds}), pe l'jng'a cel de probabilitate.
\begin{equation}
Odds: O(A) = \frac{P(A)}{P(\bar{A})} = \frac{P(A)}{1 - P(A)}.
\label{odds}
\end{equation}

\subsection{Principiul probabilistic de ierarhizare}
\subsubsection{Cazul 1/0 loss}
Presupunem c'a sistemul de RI 'intoarce ca r'aspuns la o interogare o list'a ordonat'a de documente 'si folosirea unei nota'tii binare pentru relevan't'a. Pentru o interogare \emph{q} 'si un document \emph{d}, fie $R_{d,q}$ o variabil'a aleatoare care indic'a dac'a \emph{d} este relevant 'in contextul interog'arii \emph{q}. Variabila ia valoarea 1 c'jnd documentul este relevant 'si 0 altfel.

Folosind un model probabilistic, ordinea evident'a 'in care documentele trebuie prezentate utilizatorului este dat'a de ierarhizarea documentelor dup'a probabilitatea estimat'a de relevan't'a 'in raport cu nevoia de informa'tie: $P(R_{d,q}=1|d,q)$. Vom scrie $R$ 'in loc de $R_{d,q}$. Aceasta reprezint'a temelia \emph{principiului probabilistic de ierarhizare (PRP)}:

\begin{quote}
Dac'a r'aspunsul unui sistem la fiecare cerere este o ierarhizare a documentelor din colec'tie 'in ordinea descresc'atoare a probabilita'tii de relevan't'a, unde probabilit'a'tile sunt estimate c'jt mai bine cu putin't'a pe baza datelor pe care sistemul le are la 'indem'jn'a, eficien'ta sistemului este cea mai bun'a care se poate obtine folosind aceste date.
\end{quote}

'In cel mai simplu caz al PRP, nu exist'a costuri de reg'asire sau alte motive de 'ingrijorare care s'a valorifice diferit ac'tiunile sau erorile. Se pierde un punct fie pentru 'intoarcerea unui document nerelevant, fie pentru lipsa 'intoarcerii unui document relevant. O astfel de evaluare binar'a asupra preciziei poart'a numele de \emph{1/0 loss}. Scopul este s'a se 'intoarc'a cele mai bune k rezultate posibile, pentru orice valoare k aleas'a de utilizator. PRP spune c'a documentele trebuie ierarhizate 'in ordinea descresc'atoare a $P(R=1|d,q)$.

%TODO: put bibl see man 204 - ripley
\begin{thm}
PRP este optim 'in sensul c'a minimizeaz'a pierderea a'stepat'a (sau riscul Bayes) 'in cazul 1/0 loss.
\end{thm}

Acest'a teorem'a este adev'arat'a dac'a toate probabilit'a'tile sunt corecte, ceea ce 'in practic'a este imposibil. Cu toate acestea, PRP reprezint'a o funda'tie pentru contruirea de modele de RI.

\subsubsection{Costuri de reg'asire}
S'a presupunem existen'ta unui model de costuri de reg'asire. Fie $C_1$ costul de reg'asire a unui document relevant 'si $C_0$ costul de reg'asire a uni document nerelevant. Atunci pentru un document d 'si pentru toate documentele d'' nereg'asite dac'a
\begin{equation}
C_1 \times P(R=1|d) + C_0 \times P(R=0|d) \leq C_1 \times P(R=1|d'') + C_0 \times P(R=0|d'')
\label{cost}
\end{equation}
atunci d este urm'atorul document care trebuie 'intors. Acest model asigur'a un cadru formal 'in care putem modela costurile diferen'tiale ale falselor-pozitive 'si falselor-negative.

\subsection{Modelul binar de independen't'a}
\emph{BIM} este modelul care a fost folosit cu PRP. Introduce c'jteva asum'tii simple care permit estimarea func'tiei probabilistice $P(R|d, q)$. Aici binar este echivalent cu boolean: at'jt documentele c'jt si interog'arile sunt reprezentate ca vectori binari de incident'a a termenilor. Un document $d$ este reprezentat de vectorul $\vec{x} = (x_1, ..., x_M)$, unde $x_t=1$ dac'a termenul t este prezent 'in documentul $d$ 'si $x_t=0$ altfel. 'In contextul acestei repezent'ari, multe documente pot avea aceea'si reprezentare. 'In mod similar, interogarea $q$ este reprezentat'a prin vectorul de inciden't'a $\vec{q}$. "Independen't'a" se refer'a la faptul c'a termenii sunt modela'ti asa cum apar 'in documente 'in mod independent. Modelul nu recunoa'ste niciun tip de asociere 'intre termeni. Aceast'a asump'tie este, evident, incorect'a, dar, 'in pofida acestui aspect, ofer'a rezultate satisf'ac'atoare 'in practic'a. Este asump'tia care st'a 'si la baza modelului \emph{Bayes Naiv}, 'si este 'intr-un fel echivalent'a cu asum'tia din VSM, 'in care fiecare termen reprezint'a o dimensiune ortogonala fa't'a de celelalte.

Pentru a face o strategie probabilistic'a de reg'asire precis'a, trebuie estimat modul 'in care termenii din documente contribuie la relevan't'a. Cu alte cuvinte, trebuie s'a specific'am cum frecven'ta termenilor 'intr-un document, num'arul de documente care con'tin un termen, lungimea unui document 'si alte statistici influen'teaz'a calculul relevan'tei unui document. Dup'a acest proces documentele vor fi ordonate 'in ordinea descresc'atoare a acestor probabilit'ati estimate.

Se porne'ste de la asump'tia c'a relevan'ta fiec'arui document este independent'a de relevan'ta celorlalte documente. 'In practic'a, acest lucru pune o problem'a 'in momentul 'in care sunt 'intoarse documente duplicate sau aproape duplicate. 'In contextul BIM, probabilitatea c'a un document este relevant la o interogare $P(R|d,q)$ este modelat'a via probabilitatea $P(R|\vec{x},\vec{q})$, folosind vectorii de inciden't'a. Apoi, aplic'jnd regula lui Bayes, se ob'tine:
\begin{equation}
P(R=1|\vec{x},\vec{q}) = \frac{P(\vec{x}|R=1,\vec{q})P(R=1|\vec{q})}{P(\vec{x},\vec{q})}\,\,\,\,\,\,\,\,
P(R=0|\vec{x},\vec{q}) = \frac{P(\vec{x}|R=0,\vec{q})P(R=0|\vec{q})}{P(\vec{x},\vec{q})}
\label{bayes}
\end{equation}

Aici, $P(\vec{x}|R=1,\vec{q})$ 'si $P(\vec{x}|R=0,\vec{q})$ reprezint'a probabilitatea ca dac'a un document relevant, respectiv nerelevant, este 'intors, acesta s'a aiba reprezentarea $\vec{x}$. Aceste probabilit'a'ti nu se pot calcula exact, a'sa c'a trebuie folosi'ti estimatori: statistici ale colectiei de documente sunt folosite pentru a estima aceste probabilit'a'ti. $P(R=1|\vec{q})$ 'si $P(R=0|\vec{q})$ reprezint'a probabilitatea apriori de a 'intoarce un document relevant, respectiv nerelevant, dat'a fiind interogarea $\vec{q}$. Pentru c'a un document este fie relevant fie nerelevant 'in contextul unei interog'ari, avem:
\begin{equation}
P(R=1|\vec{x},\vec{q}) + P(R=0|\vec{x},\vec{q}) = 1.
\label{prisum}
\end{equation}

\subsubsection{Derivarea unei func'tii de ierarhizare}
Dec'jt s'a estim'am $P(R=1|\vec{x},\vec{q})$ direct, deoarece intereseaz'a doar ordinea 'in care sunt 'intoarse documentele, folosim alte cantit'a'ti care sunt mai usor de calculat 'si care au ca rezultat aceea'si ordine. Putem ordona documentele dup'a 'sansele (odds) de relevan't'a, ceea ce duce la o simplificare a rela'tiei:
\begin{equation}
O(R|\vec{x},\vec{q}) = \frac{P(R=1|\vec{x},\vec{q})}{P(R=0|\vec{x},\vec{q})} = \frac{\frac{P(\vec{x}|R=1,\vec{q})P(R=1|\vec{q})}{P(\vec{x},\vec{q})}}{\frac{P(\vec{x}|R=0,\vec{q})P(R=0|\vec{q})}{P(\vec{x},\vec{q})}} = \frac{P(R=1|\vec{q})}{P(R=0|\vec{q})} \cdot \frac{P(\vec{x}|R=1,\vec{q})}{P(\vec{x}|R=0,\vec{q})}
\label{oddsrel}
\end{equation}

Termenul st'jng al expresiei din dreapta al ecua'tiei \ref{oddsrel} este constant pentru o interogare dat'a. Pentru c'a intereseaz'a doar ordinea, nu e nevoie s'a se estimeze. Trebuie estimat, 'in schimb, cel'alalt termen, lucru care pare dificil ini'tial: cum poate fie precis estimat'a probabilitatea unui 'intreg vector de inciden't'a? Pentru a face posibil'a estimarea, se face asum'tia de \emph{condi'tional-independen't'a Naive Bayes}: prezen'ta sau absen'ta unui cuv'jnt 'intr-un document este independent'a de prezen'ta sau absen'ta oric'arui alt cuv'jnt:
\begin{equation}
\frac{P(\vec{x}|R=1,\vec{q})}{P(\vec{x}|R=0,\vec{q})} = \prod\limits_{t=1}^M{\frac{P(x_t|R=1,\vec{q})}{P(x_t|R=0,\vec{q})}} 
\label{naiveb}
\end{equation}
A'sadar:
\begin{equation}
O(R|\vec{x},\vec{q}) = O(R|\vec{q}) \cdot \prod\limits_{t=1}^M{\frac{P(x_t|R=1,\vec{q})}{P(x_t|R=0,\vec{q})}}.
\label{oddsnb}
\end{equation}
Pentru c'a fiecare $x_t$ este fie 0, fie 1, putem separa termenii astfel:
\begin{equation}
O(R|\vec{x},\vec{q}) = O(R|\vec{q}) \cdot \prod\limits_{t:x_t=1}{\frac{P(x_t=1|R=1,\vec{q})}{P(x_t=1|R=0,\vec{q})}} \cdot \prod\limits_{t:x_t=0}{\frac{P(x_t=0|R=1,\vec{q})}{P(x_t=0|R=0,\vec{q})}}.
\label{oddsnb1}
\end{equation}

Fie $p_t = P(x_t=1|R=1,\vec{q})$ probabilitatea ca termenul $x_t$ s'a apara 'intr-un document relevant la interogarea dat'a 'si $u_t = P(x_t=1|R=0,\vec{q})$, probabilitatea ca termenul s'a apara 'intr-un document nerelevant. Aceste cantit'a'ti pot fi vizualizate 'in tabelul de contingen't'a urm'ator (suma pe coloane este 1):
\begin{table}[ht]
\centering
\begin{tabular}{|lc|cc|}
	\hline
	& document & relevant (R=1) & nerelevant(R=0)\\
	\hline
	termen present & $x_t = 1$ & $p_t$ & $u_t$\\
	termen absent & $x_t = 0$ & $1 - p_t$ & $1 - u_t$\\
	\hline
\end{tabular}
\label{tab:cont1}
\end{table}
 
Fac'jnd asump'tia c'a termenii care nu apar 'in interogare au probabilit'a'ti egale de apari'tie intr'un document relevant, respectiv nerelevant: $q_t = 0 \Rightarrow p_t = u_t$, vor trebui lua'ti 'in considerare doar termenii care apar 'in interogare:
\begin{equation}
O(R|\vec{x},\vec{q}) = O(R|\vec{q}) \cdot \prod\limits_{t:x_t=q_t=1}{\frac{p_t}{u_t}} \cdot \prod\limits_{t:x_t=0,q_t=1}{\frac{1-p_t}{1-u_t}}.
\label{eq:ptut}
\end{equation}
Primul produs este peste termenii interog'arii care apar 'in document 'si produsul din dreapta peste cei care nu apar.

Expresia poate fi manipulat'a prin includerea termenilor g'asi'ti 'in document 'in produsul din dreapta, dar, 'in acela'si timp, ajust'jnd produsul st'jng pentru simplificare:
\begin{equation}
O(R|\vec{x},\vec{q}) = O(R|\vec{q}) \cdot \prod\limits_{t:x_t=q_t=1}{\frac{p_t(1-u_t)}{u_t(1-p_t)}} \cdot \prod\limits_{t:q_t=1}{\frac{1-p_t}{1-u_t}}.
\label{eq:ptut1}
\end{equation}

Produsul drept este acum peste to'ti termenii interog'arii, ceea ce 'inseamn'a c'a e constant pentru o interogare, la fel ca $O(R|\vec{q})$. A'sa dar, singura cantitate care trebuie estimat'a pentru a ierarhiza documentele este cea din produsul st'jng. Putem ordona documentele 'si dup'a rezultatul logaritm'arii produsului, deoarece log este o func'tie monoton'a. Cantitatea folosit'a la ierarhizare se nume'ste \emph{valoarea statusului de reg'asire (RSV - retrieval status value)}:
\begin{equation}
RSV_d = \log{\prod\limits_{t:x_t=q_t=1}{\frac{p_t(1-u_t)}{u_t(1-p_t)}}} = \sum\limits_{t:x_t=q_t=1}{\log{\frac{p_t(1-u_t)}{u_t(1-p_t)}}}.
\label{eq:rsv}
\end{equation}

Totul se reduce la clacularea RSV. Definim $c_t$:
\begin{equation}
c_t = \log{\frac{p_t(1-u_t)}{u_t(1-p_t}} = \log{\frac{p_t}{1 - p_t}} + \log{\frac{1-u_t}{u_t}}.
\label{eq:}
\end{equation}

Termenii $c_t$ reprezint'a propor'tiile \emph{log odds} pentru termenii interog'arii. Valolare va fi 0 dac'a un termen are 'sanse egale s'a apar intr'un document relevant, respectiv nerelevant, 'si pozitiv'a dac'a este mai probabil s'a apar'a 'intr-un document relevant. Cantit'a'tile $c_t$ func'tioneaz'a ca ponderi ale termenilor 'in model, iar scorul pentru o interogare 'si un document este: $RSV_d = \sum_{x_t=q_t=1}{c_t}$. Problema r'amas'a este cum s'a se estimeze cantit'a'tile $c_t$ pentru o colec'tie de documente 'si o interogare.

\subsubsection{Estim'ari teoretice}
Urm'atorul tabel de contingen't'a prezint'a o serie de statistici ale colec'tiei, unde $df_t$ reprezint'a num'arul de documente care con'tin termenul $t$:
\begin{table}[ht]
\centering
\begin{tabular}{|lc|cc|c|}
	\hline
	& document & relevante & nerelevante & total\\
	\hline
	termen present & $x_t = 1$ & $s$ & $df_t - s$ & $df_t$\\
	termen absent & $x_t = 0$ & $S - s$ & $(N-df_t) - (S-s)$ & $N - df_t$\\
	\hline
	& total & $S$ & $N - S$ & $N$\\
	\hline
\end{tabular}
\label{tab:cont2}
\end{table}

A'sadar, $p_t = s/S$ 'si $u_t = (df_t-s)/(N-S)$ 'si
\begin{equation}
c_t = K(N,df_t,S,s) = \log{\frac{s/(S-s)}{(df_t-s)/((N-df_t)-(S-s))}}.
\label{eq:ct}
\end{equation}
Pentru a evita posibilitatea apari'tiei de zerouri (de exemplu, toate sau niciun document relevan con'tine un anumit termen) se adaug'a $\frac{1}{2}$ la fiecare dintre cele 4 cantit'a'ti 'si apoi se ajusteaz'a totalurile (N + 2). Ca urmare avem:
\begin{equation}
\hat{c_t} = K(N,df_t,S,s) = \log{\frac{(s+\frac{1}{2})/(S-s+\frac{1}{2})}{(df_t-s+\frac{1}{2})/(N-df_t-S+s+\frac{1}{2})}}.
\label{eq:ct1}
\end{equation}
Ad'ugarea valorii $\frac{1}{2}$ este o form'a simpl'a de \emph{uniformizare}.

\subsubsection{Estim'ari practice}
Sub asum'tia c'a documentele relevante reprezint'a un procent infim din colec'tie, este plauzibil'a aproximarea statisticilor pentru documentele nerelevante cu statisticile pe 'intreaga colec'tie. Ca urmare, $u_t$ (probabilitatea ca un document nerelevant s'a con'tin'a termenul t pentru o interogare) poate fi aproximat cu $df_t/N$ 'si:
\begin{equation}
\log{\frac{1-u_t}{u_t}} = \log{\frac{N-df_t}{df_t}} \approx \log{\frac{N}{df_t}}
\label{eq:assum}
\end{equation}
Rezultatul este interesant 'si prin faptul c'a furnizeaz'a o justificare teoretic'a a celei mai 'int'jlnite forme de ponderare \emph{idf} folosit'a 'in VSM.

Tehnica de aproximare din ecua'tia \ref{eq:assum} nu poate fi usor extins'a la documente relevante. Cantitatea $p_t$ poate fi estimat'a 'in mai multe moduri:
\begin{enumerate}
	\item Se poate folosi frecven'ta termenilor din documentele cunoscute deja ca fiind relevante(dac'a se cunosc - metod'a folosit'a 'in cadrul \emph{feedback-ului de relevan't'a})
	\item Se poate presupune c'a fiecare termen are s'anse egale s'a apara'a 'intr-un document relevant: $p_t = 0.5$. Aceast'a estimare este destul de slab'a. Combin'jnd aceast'a metod'a cu aproximarea lui $u_t$ de mai sus, ierarhizarea documentelor este dat'a de termenii interog'arii care apar 'in documente scala'ti cu ponderea \emph{idf}.
	\item O alt'a aproximare propus'a folose'ste statisticile apari'tiilor termenilor 'in colec'tie: $p_t = df_t/N$.
\end{enumerate}

\subsection{Okapi BM25}
Metodele probabiliste sunt unele din cele mai vechi modele formale 'in RI. 'Inc'a din 1970 erau privite ca o oportunitate pentru a pune bazele teoretice 'in RI 'si, odat'a cu "rena'sterea" modelelor probabiliste 'in lingvistica computa'tional'a 'in anii 1990, aceast'a oportunitate s'a 'intors iar metodele probabiliste reprezint'a unul dintre subiectele cele mai discutate subiecte 'in materie de RI. Ob'tinerea unor aproxim'ari rezonabile ale probabilit'a'tilor necesare pentru un model RI probabilistic este posibil'a, dar necesit'a prezum'tii majore. 'In modelul BIM acestea sunt:
\begin{itemize}
	\item o reprezentare boolean'a a documentelor, interog'arilor, relevan'tei
	\item independen'ta termenilor
	\item termenii care nu apar 'in interogare nu afecteaz'a rezultatul
	\item valorile de relevan't'a ale documentelor sunt independente
\end{itemize}

Poate c'a din cauza severit'a'tii asump'tiilor de modelare este dificil'a ob'tinerea unei performan'te mai bune. O problem'a general'a pare s'a fie c'a modelele probabiliste fie necesit'a informa'tii par'tiale de relevan't'a, fie duc la derivarea unor scheme aparent inferioare de ponderare a termenilor.

Aceast'a situa'tie s-a schimbat 'in anii 1990 c'jnd schema de ponderare \emph{BM25} a avut rezultate foarte bune 'si a 'inceput s'a fie adoptat'a de multe sisteme RI. Diferen'ta dintre sistemele RI bazate pe \emph{spatiul vectorial} 'si cele bazate pe modelul probabilistic nu este a'sa de mare; 'in ambele cazuri se construieste un sistem similar, singura diferen't'a fiind c'a scorul documentelor in contextul unei interog'ari este dat pe de-o parte de \emph{similaritatea cosinus} aplicat'a pe vectori de ponderi \emph{tf-idf}, iar pe de alt'a parte de o formul'a u'sor diferit'a motivat'a de teoria probabilit'a'tilor.

\subsubsection{Un model nebinar}
Modelul BIM a fost ini'tial proiectat pentru scurte 'inregistr'ari de cataloage 'si a func'tionat destul de bine 'in acest context, dar pentru c'aut'ari \emph{full-text} pe colec'tii mari este evident c'a un model trebuie s'a ia 'in considerare frecven'ta termenilor 'si lungimea documentelor. Schema de ponderare numit'a Okapi dup'a sistemul 'in care a fost ini'tial implementat'a, a fost proiectat'a folosind un model probabilistic sensibil la aceste tipuri de informa'tie f'ar'a s'a introduc'a prea mul'ti parametrii adi'tionali.

Cel mai simplu scor pentru un document $d$ este dat de adunarea ponderilor \emph{idf} ale termenilor interog'arii prezen'ti 'in document:
\begin{equation}
RSV_d=\sum\limits_{t \in q}{\log{\frac{N}{df_t}}}
\label{eq:okapi1}
\end{equation}
Pornind de la formula din ecua'tia \ref{eq:ct1}, 'si estim'jnd $S=s=0$ 'in absen'ta feedback-ului de relevan't'a, se ob'tine o formulare alternativ'a a idf:
\begin{equation}
RSV_d=\sum\limits_{t \in q}{\log{\frac{N-df_t+\frac{1}{2}}{df_t + \frac{1}{2}}}}
\label{eq:okapi2}
\end{equation}
Aceast'a variant'a are un comportament ciudat: dac'a un termen apare 'in peste jum'atate din documentele din colec'tie, modelul d'a o pondere negativ'a, lucru care este nedorit. 'In cazul folosirii unui \emph{stop list} acest lucru nu se 'int'jmpl'a de obicei.

Ecua'tia \ref{eq:okapi1} poate fi 'imbun'at'a'tit'a prin folosirea frecven'tei termenilor 'si a lungimii documentului:
\begin{equation}
RSV_d = \sum\limits_{t \in q}{\log{\left[\frac{N}{df_t}\right]} \cdot \frac{(k_1+1)tf_{td}}{k_1((1-b)+b\times(L_d/L_{ave}))+tf_{td}}}
\label{eq:okapi3}
\end{equation}
Aici, $tf_{td}$ este frecven'ta termenului t 'in documentul d 'si $L_d$ 'si $L_{ave}$ sunt lungimea documentului d, respectiv lungimea media a unui document din colec'tie. Variabila $k_1$ este un parametru pozitiv de ajustare care calibreaz'a scalarea frecven'tei $tf_td$: $k_1=0$ corespunde modelului binar, iar o valoare mare corespunde folosirii frecven'tei brute. $b$ este un alt parametru de calibrare ($0\leq{b}\leq{1}$) care determin'a scalarea dup'a lungimea documentului: $b=1$ presupune scalarea complet'a a ponderii termenului cu lungimea documentului 'in timp ce $b=0$ presupune lipsa normaliz'arii cu lungimea.

Dac'a interogarea este lung'a, atunci am putea folosi o ponderare similar'a pentru termenii din interogare. Acest lucru este adecvat dac'a interog'arile au lungimi de dimensiunile unui paragraf, alftel este nenecesar:
\begin{equation}
RSV_d = \sum\limits_{t \in q}{\log{\left[\frac{N}{df_t}\right]} \cdot \frac{(k_1+1)tf_{td}}{k_1((1-b)+b\times(L_d/L_{ave}))+tf_{td}} \cdot \frac{(k_3+1)tf_{tq}}{k_3+tf_{tq}}}.
\label{eq:okapi4}
\end{equation}
Aici $tf_{tq}$ este frecven'ta termenului t 'in interogarea 'si $k_3$ este un alt parametru pozitiv de calibrare care ajusteaz'a scalarea freven'tei termenilor din interogare. Parametru $b$ pentru normalizarea lungimii interog'arii este nenecesar.

Acesti parametri 'in mod ideal sunt seta'ti s'a optimizeze performan'ta pe o colec'tie de test. C'autarea valorilor care s'a maximizeze performan'ta poate fi f'acut'a manual sau automat. 'In absen'ta unor astfel de optimiz'ari, experimentele au ar'atat c'a valori bune pentru ace'sti parametrii sunt $b=0.75$ 'si $1.2\leq{k_1,k_3}\leq2$.

Formulele BM25 de ponderare a termenilor au fost folosite cu succes pe o varietate de colec'tii 'si tipuri de c'autare. Au avut o performan't'a extrem de bun'a la evalu'arile TREC 'si au fost implementate 'in multe sisteme RI.

\chapter{Metode de agregare}
\section{Borda}
\section{Agregarea Rank Distance}

\chapter{Studiu comparativ}
\section{Unelte}
\subsection{Apache Lucene}
\subsubsection{Similaritate 'si ierarhizare}
\emph{Lucene} combin'a modelul boolean (BM) cu modelul de spa'tiu vectorial (VSM): documentele care trec de BM sunt etichetate cu un scor de c'atre VSM.

%remove?
'In VSM, documentele si interog'arile sunt reprezentate ca vectori de ponderi 'intr-un spa'tiu multidimensional, unde fiecare termen din index este o dimensiune 'si ponderile sunt valorile \emph{tf-idf}. VSM nu necesit'a faptul ca ponderile s'a fie valori tf-idf, dar aceste ponderi au rezultate foarte bune 'in practic'a, 'si, ca urmare, Lucene folo'ste aceast'a abordare. Pentru un termen \emph{t} 'si un document (sau interogare) \emph{x}, tf(t,x) cre'te odat'a cu num'arul de ocuren'te ale lui t 'in x iar idf(t) descre'ste odat'a cu cre'sterea num'arului de documente din index care il con'tin pe t.

Scorul documentului d pentru interogarea q este dat de \emph{similaritatea cosinus} pentru vectorii de ponderi V(q) 'si V(d):
\[
cos-sim(q, d) = \frac{V(q)V(d)}{|V(q)||V(d)|},
\]
unde num'aratorul reprezint'a produsul scalar, iar numitorul, produsul normelor euclidiene. Ecuatia poate fi v'azut'a si ca produsul scalar dintre cei doi vectori normaliza'ti.

Lucene perfec'tioneaza'a VSM at'at 'in materie de calitate c'jt 'si de uzabilitate.
\begin{itemize}
	\item Normalizarea lui V(d) la vectorul unitate poate pune unele probleme 'in sensul c'a 'indep'arteaz'a toat'a informa'tia despre lungimea documentului. Pentru unele documente, lucrul acesta poate reprezenta o problem'a. Pentru a evita aceast'a problem'a, Lucene folose'ste un alt factor de normalizare a lungimii documentului, care normalizeaz'a vectorul la un vector mai mare sau egal dec'jt vectorul unitate: doc-len-norm(d).
	\item La indexare utilizatorii pot specifica faptul c'a unele documente sunt mai importante dec'jt altele prin asignarea unui \emph{boost} respectivelor documente. Ca urmare, scorul fiec'arui documente este multiplicat cu aceast'a valoare: doc-boost(d).
	\item Lucene este bazat pe c'jmpuri (sec'tiuni ale unui document), 'si, ca urmare, fiecare termen al unei interog'ari se aplic'a unui singur c'jmp, normalizarea vectorului se aplic'a la nivel de c'jmp, 'si se pot specifica 'si nivele de boost pentru c'jmpuri.
	\item Acela'si c'jmp poate fi ad'augat unui document 'in timpul index'arii de mai multe ori, iar, ca urmare, nivelul de boost al acelui c'jmp este dat de 'inmul'tirea nivelelor de boost ale ad'aug'arilor.
	\item La c'autare utilizatorii pot specifica nivele de boost pentru fiecare interogare, sub-interogare 'si termen al unei interog'ari.
	\item Un document poate fi relevant la o interogare cu mai mul'ti termeni f'ar'a s'a contin'a to'ti termenii prezen'ti 'in interogare, iar documentele 'in care apar mai mul'ti termeni pot fi "r'aspl'atite" printr-un factor de coordonare, care este mai mare c'jnd mai mul'ti termeni sunt prezen'ti: coord-factor(q, d).
\end{itemize}

Fac'jnd asump'tia simplificatoare c'a exist'a un singur c'jmp 'in index, \emph{formula conceptuala de scor} pentru Lucene este urm'atoarea:
\[
%calin - coord-factor
score(q, d) = coordfactor(q,d) \times queryboost(q) \times \frac{V(q) \times V(d)}{|V(q)|} \times doclennorm(d) \times docboost(d)
\]

Din aceast'a formul'a se deriveaz'a \emph{formula practic'a de scor} care este implementat'a de Lucene.

Pentru computarea eficient'a a scorului, unele componente sunt calculate 'si agregate la indexare:
\begin{itemize}
	\item Nivelul de boost pentru interogare este cunoscut c'jnd c'autarea 'incepe.
	\item Norma euclidian'a a vectorului interogare poate fi calculat'a c'jnd 'incepe c'autarea, dat fiind faptul c'a e independent'a de documentul pentru care se calculeaz'a scorul la un moment dat. Din perspectiva optimiz'arii, merit'a pus'a 'intrebarea: \emph{are rost s'a se normalizeze vectorul interog'arii, din moment ce toate scorurile vor fi multiplicate cu aceea'si valoare?} Ca urmare, ierarhia documentelor pentru o interogare dat'a nu va fi afectat'a de normalizare. Exist'a dou'a motive pentru a p'astra normalizarea:
	\begin{itemize}
		\item scorurile unui document pentru interog'ari distincte trebuie s'a fie comparabile('intr-o anumit'a m'asur'a)
		\item aplicarea normaliz'arii p'astreaz'a scorurile "'in jurul" vectorului unitate, 'impiedic'jnd astfel alterarea scorurilor din cauza limit'arii de precizie ale numerelor 'in virgul'a mobil'a
	\end{itemize}
	\item Norma pentru fiecare document doc-len-norm(d) 'si nivelul de boost doc-boost(d) sunt cunoscute la indexare. Sunt calculate 'si rezultatul 'inmul'tirii lor este salvat ca o singur'a valoare 'in index: norm(d).
\end{itemize}

'In continuare este prezentat'a formula practic'a de scor:
\[
score(q, d) = coord(q, d) \times queryNorm(q) \times \sum\limits_{t \in q}{(tf(t, d) \times idf(t)^2 \times boost(t) \times norm(field(t), d))},
\]
unde:

\begin{enumerate}
	\item \emph{tf(t, d)} este corelat cu frecven'ta termenului in document. Documentele 'in care un termen apare de mai multe ori primesc un scor mai mare pentru acel termen. Lucene implementeaz'a astfel: $tf(t, d) = \sqrt{freq}$, unde \emph{freq} reprezint'a de c'ate ori apare termenul 'in document.
	\item \emph{idf(t)} este inversul frecven'tei termenului la nivel de index. Acest lucru 'inseamna'a c'a termenii mai rari au o contribu'tie mai mare la scor. Implementeare Lucene este: $idf(t) = 1 + \log{(\frac{numDocs}{docFreq + 1})}$, unde \emph{numDocs} reprezint'a num'arul de documente din index 'si \emph{docFreq} reprezint'a num'arul de documente 'in care apare termenul.
	\item \emph{coord(q, d)} este o component'a calculat'a la momentul c'aut'arii: $coord(q, d) = \frac{overlap}{maxOverlap}$, unde \emph{overlap} reprezint'a num'arul de termeni din interogare care se reg'asesc 'in document 'si \emph{maxOverlap}, num'arul de termeni ai interog'arii.
	\item \emph{queryNorm(q)} este factorul de normalizare folosit pentru a face scorurile pentru diferite interog'ari comparabile. Acest factor nu afecteaz'a ierarhizarea documentelor din moment ce este acela'si pentru fiecare document. Implementarea implicit'a Lucene computeaz'a norma euclidian'a a vectorului ponderilor(ajustate de nivelele de boost):
	\[
	queryNorm(q) = \frac{1}{\sqrt{boost(q) \times \sum\limits_{t \in q}{(idf(t) \times boost(t))^2}}}
	\]
	\item \emph{boost(t)} reprezint'a nivelul de boost al termenului; acesta poate fi setat din sintaxa interog'arii dac'a este folosit parserul de interog'ari pus la dispozi'tie de Lucene, sau prin intermediul api-ului obiectului \emph{Query}.
\end{enumerate}


\subsection{Pachetul Benchmark}
\subsection{Apache Open Relevance Project}
\section{Construirea framework-ului}
\section{Rezultate}

\chapter{Concluzii}

\begin{thebibliography}{9}
	\bibitem{MAN}Christopher D. Manning, Prabhakar Raghavan and Hinrich Schutze, \emph{Introduction to Information Retrieval}, Cambridge University Press. 2008.
	\bibitem{IRWIKI} \url{http://en.wikipedia.org/wiki/Information_retrieval}
	%sim javadoc
\end{thebibliography}

\end{document}
