\documentclass[a4paper]{book}

%pachete folosite
\usepackage{graphicx}
\usepackage{fullpage}
\usepackage{verbatim}
\usepackage{url}
%pachete folosite

%calea catre diagrame
\graphicspath{{diagrams/}}
%calea catre diagrame

%pentru diacritice
\newcommand{\myqq}{''}
\catcode`\'=13
\newcommand{'}[1]{\ifmmode {}^\prime \noexpand#1%
\else\ifx#1a\u{a}% 
\else\ifx#1i\^{i}%
\else\ifx#1s\c{s}% 
\else\ifx#1t\c{t}%
\else\ifx#1j\^{a}%
\else\ifx#1A\u{A}% 
\else\ifx#1I\^{I}%
\else\ifx#1S\c{S}% 
\else\ifx#1T\c{T}%
\else\ifx#1J\^{A}%
\else\ifx#1'\myqq%
	\fi\fi\fi\fi\fi\fi\fi\fi\fi\fi\fi%
\fi}
%pentru diacritice

%titluri in romana 
\def\contentsname{Cuprins}
\renewcommand{\figurename}{Figura}
\renewcommand{\tablename}{Tabelul}
\renewcommand{\chaptername}{Capitolul}
\renewcommand{\bibname}{Bibliografie}
%titluri in romana


\begin{document}


%linie orizontala
\newcommand{\HRule}{\rule{\linewidth}{0.5mm}}
%linie orizontala

%data in format romanesc
\def\today{\space  \ifcase \month
\or IANUARIE\or FEBRUARIE\or mARTIE\or APRILIE\or MAI\or IUNIE\or
IULIE\or AUGUST\or SEPTEMBRIE\or OCTOMBRIE\or NOIEMBRIE\or DECEMBRIE \fi
\space  \number\year}
%data in format romanesc

%pagina de titlu
\begin{titlepage}
\begin{center}
  
%facultatea
\textsc{\LARGE Universitatea Bucure'sti}\\
\textsc{\LARGE Facultatea de matematic'a 'si informatic'a}\\[0.5cm]
\textsc{\Large Specializarea Inteligen't'a Artificial'a}\\[1.5cm]
%facultatea
 

\textsc{\Large Lucrare de diserta'tie}\\[3.0cm]
 
 
%Titlu
\HRule \\[0.5cm]
{ \huge \bfseries Metode de ierarhizare 'in Reg'asirea Informa'tiei}\\[0.4cm]
\HRule \\[4.5cm]
%Titlu
 
%autor si coordonator
\begin{minipage}{0.4\textwidth}
\begin{flushleft} \large
\emph{Autor:}\\
C'alin \textsc{Avas'ilc'ai}
\end{flushleft}
\end{minipage}
\begin{minipage}{0.4\textwidth}
\begin{flushright} \large
\emph{Coordonator:} \\
Lect. Dr. Marius \textsc{Popescu}
\end{flushright}
\end{minipage}
%autor si coordonator
 
\vfill
 
%data curenta
{\large \today}
%data curenta
 
\end{center}
\end{titlepage}
%pagina de titlu

\tableofcontents



\chapter{Introducere}
\chapter{Regasirea informa'tiei}

%\section{}
%\subsection{}
\chapter{Metode de ierarhizare}

\chapter{Metode de agregare}

\chapter{Studiu comparativ??}
\section{Unelte}
\section{Construirea framework-ului}
\section{Rezultate}

\chapter{Concluzii}

\begin{thebibliography}{9}
	\bibitem{Manning}Christopher D. Manning, Prabhakar Raghavan and Hinrich Sch�tze, \emph{Introduction to Information Retrieval}, Cambridge University Press. 2008.
	\bibitem{wikiir} \url{http://en.wikipedia.org/wiki/Information_retrieval}
\end{thebibliography}

\end{document}
